
\documentclass[11pt,a4paper]{article}

% --- Encoding / language
\usepackage[utf8]{inputenc}
\usepackage[T1]{fontenc}
\usepackage[french]{babel}

% --- Layout
\usepackage{geometry}
\geometry{margin=2.4cm}
\usepackage{setspace}
\onehalfspacing
\usepackage{microtype}

% --- Math / tables / graphics
\usepackage{amsmath,amssymb,amsthm}
\usepackage{booktabs}
\usepackage{longtable}
\usepackage{array}

% --- Links / quotes / bibliography
\usepackage{hyperref}
\hypersetup{colorlinks=true,linkcolor=black,urlcolor=black,citecolor=black}
\usepackage{csquotes}
\usepackage[numbers]{natbib}

% --- Headings
\usepackage{titlesec}
\titleformat{\section}{\large\bfseries}{\thesection.}{0.6em}{}
\titleformat{\subsection}{\normalsize\bfseries}{\thesubsection.}{0.6em}{}
\titleformat{\subsubsection}{\normalsize\itshape}{\thesubsubsection.}{0.6em}{}

% --- Title
\title{\textbf{Modéliser la Souveraineté dans la Stratégie d’Entreprise}\\
\large Livre blanc (version cabinet) — cadre quantitatif, portefeuille d’options, gouvernance}
\author{Briefing expert (brouillon de travail)\\\small{Usage : Comité Stratégie / CFO / COO / CSO}}
\date{\today}

\begin{document}
\maketitle

\vspace{-0.4cm}
\begin{center}
\small
\textbf{Question client (formulation)} : \emph{« Comment pouvons-nous modéliser la souveraineté de façon crédible dans notre stratégie ? »}
\end{center}

\section*{Résumé exécutif}

La « souveraineté » n’est pas un slogan de plus dans la grammaire stratégique : elle matérialise des \textbf{contraintes de dépendance} (technologiques, industrielles, réglementaires, financières) et des \textbf{asymétries de pouvoir} (pricing power, contrôle de standards, extraterritorialité, accès aux marchés et aux systèmes de paiement).
Pour une entreprise européenne globalisée, elle se traduit par une exigence concrète : \textbf{maintenir la capacité de décider} quand les règles du jeu changent.

Ce papier propose une approche praticable (et audit-able) : traiter la souveraineté comme un \textbf{problème de décision séquentielle} sous incertitude, où l’entreprise arbitre entre (i) performance, (ii) résilience, (iii) optionalité stratégique.
Le cœur du cadre repose sur trois briques :
\begin{enumerate}
\item \textbf{Programmation dynamique (Bellman)} : optimiser une politique d’actions sur un horizon fini, en intégrant les coûts futurs et l’optionalité.
\item \textbf{Risque extrême (CVaR)} : piloter explicitement les pertes « queue de distribution » (sanctions, ruptures, choc tarifaire durable).
\item \textbf{Robustesse distributionnelle (Wasserstein DRO)} : reconnaître que la \emph{distribution} des scénarios est elle-même incertaine et changeante.
\end{enumerate}

Le résultat n’est pas une « matrice 2x2 » : c’est une \textbf{politique conditionnelle à l’état}, un \textbf{portefeuille d’options stratégiques} (Capex/Opex, délai d’exécution, coût d’exercice \(E(t)\) variable), et un \textbf{système de gouvernance} (registre des dépendances, indicateurs, seuils de déclenchement, escalades Board/Comité/Opérations).
La méthode est compatible avec une articulation macro–micro : modèles CGE/commerce (ex. MIRAGE-e) pour scénariser les chocs \emph{macro}, et modèle décisionnel \emph{micro} pour optimiser les réponses de l’entreprise.

\section{Pourquoi « souveraineté » devient un objet de stratégie}

\subsection{Définition opératoire}
Dans ce papier, la souveraineté d’une entreprise est définie comme :
\begin{quote}
\textbf{la capacité à préserver une liberté d’action décisionnelle} face à des acteurs externes capables d’imposer des contraintes (prix, accès, normes, juridictions), \textbf{sans dégrader de façon non maîtrisée} la compétitivité de long terme.
\end{quote}

Cette définition est volontairement \textbf{économique} (et non politique) : elle se mesure à travers des dépendances et des coûts.
Elle oblige à expliciter les arbitrages : un gain d’efficacité peut augmenter une dépendance critique ; une mitigation réduit le risque mais a un coût (Capex/Opex, complexité, inertie).

\subsection{Europe : autonomie stratégique « ouverte »}
La grille européenne (autonomie stratégique \emph{ouverte}) vise une économie intégrée mais moins vulnérable aux dépendances critiques.
Des prises de position politiques récentes (Draghi sur la compétitivité, Letta sur l’approfondissement du marché unique) soutiennent l’idée que les choix industriels, technologiques et financiers deviennent stratégiques pour l’Union \citep{letta2024,draghi2024}.

\section{Optionalité stratégique : le concept, puis l’outil}

\subsection{Ce qu’on entend par « optionalité »}
L’optionalité stratégique est la valeur de \textbf{disposer de choix futurs} :
\begin{itemize}
\item \textbf{Retarder} une décision irréversible (attendre une information utile).
\item \textbf{Accélérer} quand une fenêtre se ferme (sanction, tarif, standard).
\item \textbf{Basculer} vers une alternative (fournisseur, techno, juridiction).
\item \textbf{Sortir} d’un couloir de dépendance (exit industriel/contractuel).
\end{itemize}
Elle s’oppose à une stratégie purement « statique » : l’objectif n’est plus seulement d’optimiser une trajectoire centrale, mais de \textbf{préserver des bifurcations} dans un monde non stationnaire.

\subsection{Fondements scientifiques : options réelles et décision séquentielle}
Deux fondations se combinent :
\begin{enumerate}
\item \textbf{Options réelles} : les investissements sous incertitude ont une valeur de flexibilité (attente, expansion, abandon, switching). L’incertitude \emph{augmente} souvent la valeur de la flexibilité \citep{dixitpindyck1994}.
\item \textbf{Programmation dynamique} : Bellman formalise que l’optimalité se construit par récurrence : les décisions futures optimales conditionnent la valeur des décisions présentes \citep{bellman1957}.
\end{enumerate}

\subsection{De la théorie à l’applicable : coût d’exercice \(E(t)\), Capex/Opex, et contraintes de délai}
Pour être praticable en entreprise, l’optionalité doit être « comptable » :
\begin{itemize}
\item \textbf{Capex} (ex. migration industrielle, re-platforming, qualification fournisseur).
\item \textbf{Opex récurrent} (ex. multi-sourcing, run parallèle, assurance, redondance).
\item \textbf{Coût d’exercice \(E(t)\)} : le coût réel d’activer l’option dépend du temps (fenêtres contractuelles, disponibilité d’équipes, backlog technique), et peut être discontinu.
\item \textbf{Contrainte de sortie à 3 ans} : une dépendance contractuelle impose une inertie ; la stratégie doit optimiser \emph{pendant} la période de verrouillage, pas seulement « après ».
\end{itemize}
On ne cherche pas à « éliminer » le risque : on cherche à dimensionner une \textbf{prime de souveraineté} (budget d’optionalité) au bon niveau.

\section{Cadre méthodologique : jeu dynamique, risque extrême, robustesse}

\subsection{Le problème : dépendance à un fournisseur sous choc tarifaire / pricing power}
Cas fil rouge : une entreprise EU dépend d’une technologie US ; un choc (tarif +100\% et/ou hausse unilatérale) modifie structurellement les coûts.
L’entreprise dispose de plusieurs actions (attendre, couvrir/hedge, investir dans une alternative, accélérer, sortir). La \textbf{Nature} représente le régime exogène (tarifs, politique, réglementation, marché), potentiellement adversarial (au sens « worst-case »).

\subsection{Programmation dynamique (Bellman)}
On définit un état \(s_t\) (régime, progression de sortie, flags d’investissement/couverture/exit) et une action \(a_t\).
La fonction de valeur :
\[
V_t(s) = \min_{a \in \mathcal{A}(s)} \ \rho_t\Big( L(s,a,\xi_{t+1}) + \beta V_{t+1}(s') \Big),
\]
où \(L\) est la perte/coût, \(\beta\) un facteur d’actualisation, \(\rho_t\) un opérateur de risque (ci-dessous), et \(s'\) l’état suivant.

\subsection{CVaR : piloter les scénarios de queue}
La CVaR capture la perte moyenne au-delà d’un quantile \(\alpha\) : plutôt que « l’espérance », elle vise les \emph{mauvais} cas \citep{rockafellarryasev2000}.
Dans les sujets de souveraineté, ce sont précisément les queues (sanctions, ruptures, escalade) qui importent.

\subsection{Ambiguïté et DRO Wasserstein}
En pratique, les probabilités de transition (ex. \(p_{01}\), \(p_{10}\) entre régimes) sont mal connues et instables.
La DRO (distributionally robust optimization) traite l’incertitude sur la distribution elle-même, en optimisant contre un ensemble de distributions proches d’une référence, mesuré par une distance de Wasserstein \citep{esfahani-kuhn2018}.

\section{Portefeuille souveraineté : multi-dépendances, multi-critères, corrélations}

\subsection{De la dépendance unique au portefeuille}
Une entreprise ne gère pas une dépendance isolée, mais un \textbf{registre} de dépendances critiques.
Le passage au portefeuille implique : corrélations de chocs, contraintes de capacité, ordonnancement, arbitrages multi-critères \citep{roadmapDoc}.

\subsection{Frontière de Pareto et décisions gouvernables}
Le modèle doit produire des objets gouvernables : matrice action–résultats, ensemble des actions non dominées, politique recommandée + conditions de bascule (triggers).

\subsection{Mean-field : quand le portefeuille devient grand}
Quand \(N\) devient large, l’espace d’états explose.
Une approximation « mean-field » permet de raisonner sur une distribution \(\mu\) des états \citep{mfgSurvey,bellmanWassMFG}.

\section{Gouvernance : rendre la souveraineté pilotable (et audit-able)}

\subsection{Du modèle au système : registre, indicateurs, seuils, escalades}
Le cadre proposé reprend une architecture en trois couches (Board / Comité / Opérations), un calendrier de revue, et des seuils de déclenchement \citep{bellmanWassMFG,implementationRoadmap}.

\subsection{Seuils et cadence (exemples)}
Exemples : \(|\Delta \varepsilon| > 0.05\), \(|\Delta p_{01}| > 0.03\), \(|\Delta \mu| > 0.10\).
Escalades : détérioration de \(V_0\) (Comité/Board) ou changement de politique \citep{bellmanWassMFG}.

\section{Articulation macro–micro : éviter le faux débat}
Un modèle macro (CGE/commerce) sert à scénariser (tarifs, réallocation, termes de l’échange) ; un modèle micro sert à décider sous contraintes (délai, cash, capacité).
L’articulation optimale est hybride \citep{mirageDoc}.

\section{Limites, robustesse, et bonnes pratiques}
Ce cadre est robuste (DP + CVaR + DRO), mais n’élimine pas le model risk : calibrations conservatrices, backtests, revue quant, gouvernance de modèle (versioning, audit trail).

\appendix
\section{Annexe A — CVaR (rappel)}
\(
\mathrm{CVaR}_\alpha(X)=\min_{\eta}\left[\eta+\frac{1}{1-\alpha}\mathbb{E}(X-\eta)_+\right]
\) \citep{rockafellarryasev2000}.

\section{Annexe B — DRO Wasserstein (intuition)}
On protège l’optimisation contre des distributions alternatives « proches » selon un budget \(\varepsilon(t)\).
\(\varepsilon(t)\) est interprétable comme \textbf{prime de souveraineté} \citep{esfahani-kuhn2018}.

\bibliographystyle{plainnat}
\bibliography{refs_v5}
\end{document}
