
\documentclass[11pt,a4paper]{article}

% ---------- Langue / encodage ----------
\usepackage[utf8]{inputenc}
\usepackage[T1]{fontenc}
\usepackage[french]{babel}

% ---------- Mise en page ----------
\usepackage{geometry}
\geometry{margin=2.35cm}
\usepackage{setspace}
\onehalfspacing
\usepackage{microtype}

% ---------- Math / tableaux ----------
\usepackage{amsmath,amssymb,amsthm}
\usepackage{booktabs}
\usepackage{longtable}
\usepackage{array}

% ---------- Citations / liens ----------
\usepackage{hyperref}
\hypersetup{colorlinks=true,linkcolor=black,urlcolor=black,citecolor=black}
\usepackage{csquotes}
\usepackage[numbers]{natbib}

% ---------- Titres ----------
\usepackage{titlesec}
\titleformat{\section}{\large\bfseries}{\thesection.}{0.6em}{}
\titleformat{\subsection}{\normalsize\bfseries}{\thesubsection.}{0.6em}{}
\titleformat{\subsubsection}{\normalsize\itshape}{\thesubsubsection.}{0.6em}{}

% ---------- Titre ----------
\title{\textbf{Modéliser la souveraineté dans la stratégie d’entreprise}\\
\large Note de position (ton cabinet) — optionalité stratégique, portefeuille, gouvernance, outils quantitatifs}
\author{Briefing expert (brouillon de travail)\\\small{Destinataires : CSO / CFO / COO / Comité Exécutif}}
\date{\today}

\begin{document}
\maketitle

\vspace{-0.3cm}
\begin{center}
\small
\textbf{Question client (formulation)} : \emph{« Comment modéliser la souveraineté de façon crédible dans notre stratégie, au-delà des matrices 2x2 ? »}
\end{center}

\section*{Résumé exécutif (lecture C-level)}

\textbf{Constat.} Pour une entreprise européenne globalisée, la souveraineté ne se réduit pas à une posture. Elle se matérialise par des \textbf{dépendances} (technologies critiques, fournisseurs, cloud, standards, juridictions, systèmes de paiement) et des \textbf{asymétries de pouvoir} (pricing power, contrôle des routes commerciales, extraterritorialité, sanctions, restrictions d’export).
Ces facteurs évoluent dans le temps ; ils ne se traitent pas correctement par des scénarios statiques.

\textbf{Proposition.} Nous proposons un cadre \textbf{quantitatif et gouvernable} qui transforme la souveraineté en discipline d’allocation (capex/opex) et de design opérationnel, structurée autour de l’\textbf{optionalité stratégique} :
la capacité à préserver et activer des choix futurs (bascules technologiques, multi-sourcing, redesign contractuel, relocalisation, découplage partiel), sous contrainte de délais et d’inerties.

\textbf{Briques scientifiques.} Le cadre repose sur trois outils éprouvés en décision et en gestion des risques :
(i) \textbf{programmation dynamique} (Bellman) pour la décision séquentielle \citep{bellman1957},
(ii) \textbf{CVaR} pour piloter les pertes de queue (événements rares à impact majeur) \citep{rockafellarryasev2000},
(iii) \textbf{optimisation robuste distributionnelle} (Wasserstein DRO) pour se protéger contre l’erreur de modèle et l’instabilité des probabilités \citep{esfahani-kuhn2018}.

\textbf{Sorties décisionnelles.} On obtient : (a) une \textbf{politique} d’actions conditionnelle à l’état (quoi faire, quand, et dans quelles conditions), (b) une estimation de la \textbf{prime de souveraineté} (budget d’optionalité), (c) des \textbf{triggers} opérationnels (seuils) qui déclenchent recalcul, accélération, ou escalade de gouvernance.

\textbf{Articulation Europe / macro-finance.} Les recommandations Draghi et Letta sur la compétitivité et l’avenir du marché unique éclairent le diagnostic : réduire les dépendances critiques et financer l’investissement de long terme \citep{draghiEC,lettaReport}.
Le cadre s’articule naturellement avec des modèles macro de commerce (CGE), dont MIRAGE (CEPII), pour scénariser les chocs, puis optimiser la réponse micro de l’entreprise \citep{mirageSite,mirageDoc2026}.

\section{Pourquoi le sujet est structurel (et pas conjoncturel)}

\subsection{Europe : compétitivité, marché unique, autonomie stratégique}
Deux textes récents structurent le débat européen.
Le rapport Letta \emph{Much More Than a Market} insiste sur l’approfondissement du marché unique et le traitement d’angles morts (énergie, capitaux, défense, innovation) \citep{lettaReport}.
La Commission a publié une page dédiée au rapport Draghi sur l’avenir de la compétitivité européenne \citep{draghiEC}, et le texte de présentation de Draghi met l’accent sur le « nouveau paysage » pour l’Europe \citep{draghiAddress}.


\subsection{Puissances intermédiaires et coalitions}
Dans un monde de rivalité entre grandes puissances, la capacité des « middle powers » à constituer des coalitions à géométrie variable devient un paramètre de stabilité.
Dans son adresse spéciale à Davos 2026, Mark Carney appelle explicitement à une coordination accrue des puissances intermédiaires face à la montée du \emph{hard power} et de l’intimidation économique \citep{carneyWEF2026}.
Pour l’entreprise européenne, la traduction est directe : les architectures de dépendance doivent être pensées en cohérence avec des blocs, des régimes de sanctions et des standards en compétition.


\subsection{Réalité opérationnelle : deals, tarifs, extraterritorialité}
Pour l’entreprise, la souveraineté apparaît sous forme de \textbf{décisions imposées} : hausse unilatérale de prix, restrictions d’export, changement de standard, obligation de localisation de données, exigences de contrôle d’accès, etc.
Dans notre fil rouge, un fournisseur US utilise son pouvoir de marché ; un choc tarifaire (ex. +100\%) et/ou contractuel reconfigure le P\&L. Le sujet n’est pas \emph{si} cela peut arriver, mais \emph{comment} l’entreprise arbitre le coût d’une mitigation aujourd’hui contre le coût d’une dépendance demain.

\section{Optionalité stratégique : ce que c’est, et ce que ce n’est pas}

\subsection{Définition opératoire}
Nous appelons \textbf{optionalité stratégique} la valeur de disposer de choix futurs activables, même si leur activation n’est pas certaine.
Ce n’est pas une posture ; c’est un portefeuille d’options concrètes : qualifier un fournisseur alternatif, rendre une architecture portable, négocier des clauses de sortie, maintenir une capacité de double-run, etc.

\subsection{Fondations scientifiques}
Deux fondations se combinent :
\begin{itemize}
\item \textbf{Options réelles} : un investissement sous incertitude porte une valeur de flexibilité (attente, expansion, abandon, switching) \citep{dixitpindyck1994}.
\item \textbf{Décision séquentielle} : la valeur d’une action aujourd’hui dépend des actions disponibles demain (principe d’optimalité de Bellman) \citep{bellman1957}.
\end{itemize}

\subsection{Rendre l’optionalité « CFO-compatible » : Capex, Opex, coût d’exercice \(E(t)\)}
L’optionalité doit être budgétable :
\begin{itemize}
\item \textbf{Capex} (migration industrielle, re-platforming, qualification, outillage).
\item \textbf{Opex récurrent} (multi-sourcing, run parallèle, assurance, redondance).
\item \textbf{Coût d’exercice \(E(t)\)} : le coût réel d’activer l’option dépend du temps (fenêtres contractuelles, capacité des équipes, contraintes d’approvisionnement). Il est souvent \emph{variable} et parfois \emph{discontinu}.
\item \textbf{Contrainte temporelle} : si une sortie opérationnelle n’est possible qu’en 3 ans (contrat ou inertie industrielle), la politique optimale doit couvrir la phase de verrouillage et la phase de sortie.
\end{itemize}

\section{Le cadre quantitatif : DP + CVaR + Wasserstein (en clair, puis en équations)}

\subsection{Intuition (niveau CSO/CFO)}
On modélise la stratégie comme une \textbf{suite de choix} sur un horizon fini (périodes), face à une Nature qui fait évoluer le régime (tarifs, sanctions, macro, réglementation). On veut :
\begin{itemize}
\item éviter les décisions myopes (qui optimisent le court terme mais enferment),
\item piloter les mauvais cas (pertes extrêmes),
\item être robuste au fait que les probabilités sont mal connues et changent.
\end{itemize}

\subsection{Formulation minimale}
Soit \(s_t\) un état (régime, progression de sortie, flags d’investissement/hedge/exit, etc.), \(a_t\) une action, \(\xi\) un aléa (régime futur).
\[
V_t(s) = \min_{a \in \mathcal{A}(s)} \ \rho_t\Big( L(s,a,\xi_{t+1}) + \beta V_{t+1}(s') \Big),
\]
où \(L\) est le coût/« loss », \(\beta\) une actualisation, et \(\rho_t\) un opérateur de risque. Ici :
\[
\rho_t(\cdot)=\sup_{\mathbb{P} \in \mathcal{B}_W(\mathbb{P}_0,\varepsilon(t))}\mathrm{CVaR}_\alpha^{\mathbb{P}}(\cdot),
\]
c’est-à-dire : CVaR au niveau \(\alpha\), \emph{au pire} sur un ensemble de distributions proches de \(\mathbb{P}_0\) selon une boule de Wasserstein de rayon \(\varepsilon(t)\) \citep{rockafellarryasev2000,esfahani-kuhn2018}.

\subsection{\(\varepsilon(t)\) : interprétation « prime de souveraineté »}
Le paramètre \(\varepsilon(t)\) pilote l’aversion à l’ambiguïté : plus \(\varepsilon(t)\) est grand, plus la politique devient prudente face à l’incertitude de modèle. Dans un langage exécutif, \(\varepsilon(t)\) agit comme une \textbf{prime de souveraineté} : le prix accepté pour se protéger contre des trajectoires plausibles non anticipées.

\section{Du cas unique au portefeuille (N dépendances, K parties, horizon fini)}

\subsection{Portefeuille de dépendances}
Une entreprise ne gère pas une dépendance unique ; elle gère un \textbf{portefeuille} (IT, composants, données, paiements, juridictions, talents). Le portefeuille introduit :
corrélations, contraintes de capacité (cash, équipes), ordonnancement, et arbitrage multi-critères.

\subsection{Jeu à K parties et Nature}
On peut généraliser à \(K\) parties : entreprise, fournisseurs dominants, régulateurs, concurrents, coalitions, et Nature.
L’intérêt opérationnel n’est pas de « résoudre » un jeu théorique abstrait, mais de formaliser les espaces d’actions, les asymétries, et les contraintes de temps — pour produire des politiques et des triggers gouvernables.

\subsection{Scalabilité : mean-field}
Quand \(N\) devient grand, on passe d’un suivi « cas par cas » à une approche par \textbf{distribution d’états} (mean-field) : on suit une mesure \(\mu_t\) qui agrège l’exposition, et on fait dépendre la décision d’une entreprise des propriétés globales du portefeuille (congestion, effets de réseau, rareté de capacités).
C’est une voie réaliste pour industrialiser le modèle \citep{carmonaDelarue2018}.

\section{Gouvernance : rendre le quant actionnable (et audit-able)}

\subsection{Le principe : une discipline, un système}
Le modèle vaut s’il est intégré à un système de gouvernance : registre, indicateurs, triggers, décisions, audit trail.
C’est exactement l’orientation décrite dans la documentation de framework et la roadmap d’implémentation (Operating System) \citep{internalFramework,internalRoadmap}.

\subsection{Operating Model (proposition)}
\textbf{Board / Comité Exécutif} : fixe l’appétence au risque souverain, valide la prime de souveraineté (Capex/Opex), arbitre les transformations irréversibles.\\
\textbf{Comité Stratégie (CSO/CFO/COO)} : pilote le portefeuille, valide les règles de triggers, suit les frontières Pareto (performance vs résilience vs optionalité).\\
\textbf{Opérations (BU/IT/Achats/Risk)} : collecte des métriques, exécute les plans, remonte signaux, maintient le registre, documente les décisions.

\subsection{Triggers (exemples) et cadence}
Exemples de triggers : variation significative de \(\varepsilon(t)\), bascule de politique optimale, dégradation de la valeur \(V_0\) au-delà d’un seuil, rupture d’hypothèses (capacité fournisseur, délais, contraintes réglementaires).
La cadence doit être adaptée au cycle : \emph{mensuel} en normal, \emph{hebdomadaire} en stress, \emph{événementiel} en crise.

\section{Macro \& micro : articulation avec MIRAGE (et pourquoi c’est utile)}

Les modèles CGE (commerce) apportent une capacité de scénarisation macro (tarifs, réallocation, termes de l’échange).
MIRAGE, développé par le CEPII depuis 2001, est explicitement positionné pour l’analyse de politiques commerciales \citep{mirageSite,mirageJRC}.
L’articulation recommandée : \textbf{macro pour scénariser}, \textbf{micro pour décider}, \textbf{gouvernance pour exécuter}.

\section{Limites et exigences de crédibilité}

Ce cadre ne prétend pas prédire la géopolitique. Il structure la décision sous incertitude, explicite les arbitrages, et rend le pilotage audit-able.
La crédibilité vient de : (i) calibrations conservatrices, (ii) backtests sur événements historiques, (iii) gouvernance de modèle (versioning, revue, traçabilité), (iv) transparence sur les hypothèses.

\appendix
\section{Annexe A — CVaR (rappel en une ligne)}
\(
\mathrm{CVaR}_\alpha(X)=\min_{\eta}\left[\eta+\frac{1}{1-\alpha}\mathbb{E}(X-\eta)_+\right]
\) \citep{rockafellarryasev2000}.


\bibliographystyle{plainnat}
\bibliography{refs_v6}
\end{document}
